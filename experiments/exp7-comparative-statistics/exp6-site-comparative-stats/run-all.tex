\documentclass{article}\usepackage[]{graphicx}\usepackage[]{color}
% maxwidth is the original width if it is less than linewidth
% otherwise use linewidth (to make sure the graphics do not exceed the margin)
\makeatletter
\def\maxwidth{ %
  \ifdim\Gin@nat@width>\linewidth
    \linewidth
  \else
    \Gin@nat@width
  \fi
}
\makeatother

\definecolor{fgcolor}{rgb}{0.345, 0.345, 0.345}
\newcommand{\hlnum}[1]{\textcolor[rgb]{0.686,0.059,0.569}{#1}}%
\newcommand{\hlstr}[1]{\textcolor[rgb]{0.192,0.494,0.8}{#1}}%
\newcommand{\hlcom}[1]{\textcolor[rgb]{0.678,0.584,0.686}{\textit{#1}}}%
\newcommand{\hlopt}[1]{\textcolor[rgb]{0,0,0}{#1}}%
\newcommand{\hlstd}[1]{\textcolor[rgb]{0.345,0.345,0.345}{#1}}%
\newcommand{\hlkwa}[1]{\textcolor[rgb]{0.161,0.373,0.58}{\textbf{#1}}}%
\newcommand{\hlkwb}[1]{\textcolor[rgb]{0.69,0.353,0.396}{#1}}%
\newcommand{\hlkwc}[1]{\textcolor[rgb]{0.333,0.667,0.333}{#1}}%
\newcommand{\hlkwd}[1]{\textcolor[rgb]{0.737,0.353,0.396}{\textbf{#1}}}%
\let\hlipl\hlkwb

\usepackage{framed}
\makeatletter
\newenvironment{kframe}{%
 \def\at@end@of@kframe{}%
 \ifinner\ifhmode%
  \def\at@end@of@kframe{\end{minipage}}%
  \begin{minipage}{\columnwidth}%
 \fi\fi%
 \def\FrameCommand##1{\hskip\@totalleftmargin \hskip-\fboxsep
 \colorbox{shadecolor}{##1}\hskip-\fboxsep
     % There is no \\@totalrightmargin, so:
     \hskip-\linewidth \hskip-\@totalleftmargin \hskip\columnwidth}%
 \MakeFramed {\advance\hsize-\width
   \@totalleftmargin\z@ \linewidth\hsize
   \@setminipage}}%
 {\par\unskip\endMakeFramed%
 \at@end@of@kframe}
\makeatother

\definecolor{shadecolor}{rgb}{.97, .97, .97}
\definecolor{messagecolor}{rgb}{0, 0, 0}
\definecolor{warningcolor}{rgb}{1, 0, 1}
\definecolor{errorcolor}{rgb}{1, 0, 0}
\newenvironment{knitrout}{}{} % an empty environment to be redefined in TeX

\usepackage{alltt}
\usepackage{fancyhdr}
\usepackage[margin=2.0cm]{geometry}%rounded up from 1.87, just to be safe...
\usepackage{parskip}
\usepackage{float}
%\usepackage{times} %make sure that the times new roman is used
\usepackage{mathptmx}

\usepackage{blindtext}
\title{Exploration of the Decentralization Method.}
\date{October 2019}
\author{Hallett group}


%      ------ Format Stuff ---------
\newlength{\itemdist}
\setlength{\itemdist}{0.05ex}
\newlength{\headdist}
\setlength{\headdist}{0.04ex}

\newcommand{\R}{\mathbb{R}}
\IfFileExists{upquote.sty}{\usepackage{upquote}}{}
\begin{document}
%\SweaveOpts{concordance=TRUE}
%\pagestyle{fancy}

\maketitle

We explore the so-called {\em decentralization method} that tries to adjust observed
count matrices in comparative metagenomic analyses.

\subsection{Introduction}

The basic observation here is that current metagenomic studies are relativistic in nature.
A sample is taken from two more more sites. 
Although the absolute number of organisms in two samples of the same size from
the two sites might differ, 
most modern -omic technologies require a specific amount of starting material
For example, a specific number of micrograms of cDNA will be required for next
generation sequencing, or a specific number of micrograms of protein will be required
for mass spectrometry.
Therefore, biomaterial harvested from the sites is either amplified or distilled to
this level, meaning all subsequent analyses are relativistic in nature.
Specifically, each samples produces a raw number of observations and these numbers
may differ across the sampling sites.
However, they cannot be attributed to biological differences (eg the number of micro-organisms
per microlitre); rather due to technological variance (eg differences in calibration 
of the machine or sample preparation).

Each site ina metagenomic study essentially provides a frequency vector across all organisms 
that can be used to compare the relative concentrations between organisms within that site.
However it is problematic to compare the abundance of two organisms between sites (without
some a priori information regarding the normalized absolute counts).

An issue that arises in several studies including our study of two sites of the Barbados reef is 
that  a very small number of organisms are greatly increased at one of the two sites but the 
remainder of the organismal population does not change significantly.
The frequency vector across all organisms between the two sites can suggest dramatic global changes.
A common pattern is that components of the frequency vector for the site whose population has increased greatly for a handful of organisms is almost every where else smaller than the components of the frency vector of the second population. 
In other words, the first population's distribution has become centralized.
We discuss a strategy for removing decentralizing these distributions in order to better detect true absolulte differences between the populations at each site.



We will need the functions and datasets from the Barbados reef project later in our analysis.
\begin{knitrout}
\definecolor{shadecolor}{rgb}{0.969, 0.969, 0.969}\color{fgcolor}\begin{kframe}
\begin{alltt}
\hlkwd{options}\hlstd{(}\hlkwc{warn} \hlstd{=} \hlopt{-}\hlnum{1}\hlstd{)}
\hlkwd{setwd}\hlstd{(}\hlstr{"~/repo/reefmicrobiome/experiments/exp5-decentralization"}\hlstd{)}
\hlkwd{library}\hlstd{(xtable)}
\hlkwd{source}\hlstd{(}\hlstr{"~/repo/reefmicrobiome/src/functions.R"}\hlstd{)}
\hlkwd{load}\hlstd{(}\hlstr{"~/repo/reefmicrobiome/data/tree.latest"}\hlstd{)}
\hlstd{tree[}\hlnum{1}\hlstd{,]}
\end{alltt}
\begin{verbatim}
##   name tax_id parent    rank embl_code division_id  br_bel   br_may
## 1 root      1     NA no rank                     8 4708322 10181105
##   br_bel_frac br_may_frac Local.Freq.Bel Local.Freq.May Glob.Freq.Bel
## 1           0           0             NA             NA             1
##   Glob.Freq.May DeltaFreq Multinom Polarity Polarity.Adj path
## 1             1         0        0       NA           NA root
\end{verbatim}
\end{kframe}
\end{knitrout}


\end{document}
